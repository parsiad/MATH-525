%% LyX 2.2.1 created this file.  For more info, see http://www.lyx.org/.
%% Do not edit unless you really know what you are doing.
\documentclass[english,12pt]{article}
\usepackage[T1]{fontenc}
\usepackage[latin9]{inputenc}
\usepackage{color}
\usepackage{babel}
\usepackage{mathtools}
\usepackage{amssymb}
\usepackage{esint}
\usepackage[unicode=true,pdfusetitle,
 bookmarks=true,bookmarksnumbered=false,bookmarksopen=false,
 breaklinks=false,pdfborder={0 0 1},backref=false,colorlinks=true]
 {hyperref}
\usepackage{breakurl}

\makeatletter
%%%%%%%%%%%%%%%%%%%%%%%%%%%%%% User specified LaTeX commands.
\usepackage[margin=1in]{geometry}

\makeatother

\begin{document}

\title{Math 525: Assignment 3 Solutions}

\date{\date{}}
\maketitle
\begin{enumerate}
\item ~
\begin{enumerate}
\item As per the hint, $(-\infty,x]^{c}=(x,\infty)=(x,x+2)\cup(x+1,x+3)\cup\cdots$
is just a countable union of open intervals. Therefore, is generated
by $\{(-\infty,x]\colon x\in\mathbb{R}\}\subset\sigma(\mathcal{G})$.
Applying $\sigma(\cdot)$ to both sides of this containment,
\[
\mathcal{B}(\mathbb{R})=\sigma(\left\{ (-\infty,x]\colon x\in\mathbb{R}\right\} )\subset\sigma(\sigma(\mathcal{G}))=\sigma(\mathcal{G}).
\]
Moreover, note that $\mathcal{G}\subset\mathcal{B}(\mathbb{R})$ since
any open interval is a Borel set. Once again applying $\sigma(\cdot)$
to both side of this containment,
\[
\sigma(\mathcal{G})\subset\sigma(\mathcal{B}(\mathbb{R}))=\mathcal{B}(\mathbb{R}).
\]
\item We must establish the three properties of a $\sigma$-algebra. (i)
$\emptyset\in\mathcal{M}$ since $f^{-1}(\emptyset)=\emptyset\in\mathcal{B}(\mathbb{R})$.
(ii) Let $B\in\mathcal{M}$. Then, $f^{-1}(B^{c})=(f^{-1}(B))^{c}\in\mathcal{B}(\mathbb{R})$
and hence $B^{c}\in\mathcal{M}$. (iii) Let $B_{1},B_{2},\ldots\in\mathcal{M}$.
Then, $f^{-1}(\cup_{n\geq1}B_{n})=\cup_{n\geq1}f^{-1}(B_{n})\in\mathcal{B}(\mathbb{R})$
and hence $\cup_{n\geq1}B_{n}\in\mathcal{M}$.
\item This is actually just a special case of Lindel�f's lemma. If you've
seen it, you'll know, if you haven't you just might in a future analysis
class.
\item Applying $\sigma(\cdot)$ to both sides of the containment $\mathcal{G}\subset\mathcal{M}$,
\[
\mathcal{B}(\mathbb{R})=\sigma(\mathcal{G})\subset\sigma(\mathcal{M})=\mathcal{M}.
\]
\end{enumerate}
\item A discrete random variable $X$ is one for which we can find a countable
set $\{x_{n}\}_{n}$ satisfying $\sum_{n\geq1}\mathbb{P}(\{X=x_{n}\})=1$.
Moreover, we showed in class that $\mathbb{P}(\{X=x_{n}\})=F(x_{n})-F(x_{n}-)$.
Combining these two facts gives the desired result.
\item ~
\begin{enumerate}
\item Note that for $x>0$,
\[
F(x)=\int_{0}^{x}\lambda e^{-\lambda x}dx=1-e^{-\lambda x}.
\]
Therefore, for $0<y<1$,
\[
F^{-1}(y)=-\frac{\ln(1-y)}{\lambda}.
\]
Therefore, we can, without loss of generality, define $X=F^{-1}(Y)$
with $F^{-1}$ as above. Note that we have ignored the events $\{Y=0\}$
and $\{Y=1\}$ since these occur with probability zero.
\item We saw in class that $X$ has distribution function $F$.
\end{enumerate}
\item Let $Y=\min\{X_{1},\ldots,X_{n}\}$. We saw in class that when $Y$
is integrable,
\[
\mathbb{E}Y=\sum_{n\geq1}\mathbb{P}\left(\left\{ Y\geq n\right\} \right).
\]
Therefore,
\[
\mathbb{E}Y=\sum_{n\geq1}\mathbb{P}\left(\left\{ \min\left\{ X_{1},\ldots,X_{m}\right\} \geq n\right\} \right)=\sum_{n\geq1}\mathbb{P}\left(\left\{ X_{1},\ldots,X_{m}\geq n\right\} \right)=\sum_{n\geq1}\mathbb{P}\left(\left\{ X_{1}\geq n\right\} \right)^{m}
\]
as desired (with the sum being on the right-hand side being infinite
when $Y$ is not integrable).
\item The inequality $|X|^{2}\leq|X|+1$ shows that $X^{2}$ is integrable
whenever $X$ is (we saw this in class).
\end{enumerate}

\end{document}
