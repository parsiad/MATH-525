%% LyX 2.2.1 created this file.  For more info, see http://www.lyx.org/.
%% Do not edit unless you really know what you are doing.
\documentclass[english,12pt]{article}
\usepackage[T1]{fontenc}
\usepackage[latin9]{inputenc}
\usepackage{units}
\usepackage{amsmath}

\makeatletter
%%%%%%%%%%%%%%%%%%%%%%%%%%%%%% User specified LaTeX commands.
\usepackage[margin=1in]{geometry}

\makeatother

\usepackage{babel}
\begin{document}

\title{Math 525: Assignment 7}

\date{\date{}}
\maketitle
\begin{enumerate}
\item Let $X_{n},Y_{n}\sim B(n,\lambda/n)$ be independent Binomial random
variables. Let $Z_{n}=X_{n}-Y_{n}$. Show that $Z_{n}$ converges
in distribution to some random variable $Z$ with characteristic function
\[
\phi(t)=e^{2\lambda\left(\cosh t-1\right)}.
\]
Actually, this is called a Skellam random variable (look it up on
Wikipedia).
\item Ten numbers $X_{1},\ldots,X_{10}$ are rounded to the nearest integer
$[X_{1}],\ldots,[X_{10}]$ and then summed $[X_{1}]+\ldots+[X_{10}]$.
Assume the errors from rounding $Y_{j}=X_{j}-[X_{j}]$ are independent
and uniformly distributed in $[-\nicefrac{1}{2},\nicefrac{1}{2}]$.
Use the CLT to determine the approximate probability that the error
\[
\left|X_{1}+\cdots+X_{10}-\left([X_{1}]+\cdots+[X_{10}]\right)\right|
\]
is no greater than one.
\item Let $(X_{n})_{n\geq0}$ be a Markov chain. Show that $((X_{n},X_{n+1}))_{n\geq0}$
is also a Markov chain.
\item ~
\begin{enumerate}
\item Show that if $P$ is a transition matrix, then $P^{2}$ is also a
transition matrix. Use this to conclude, by induction, that $P^{n}$
is a transition matrix for any $n$.
\item A matrix $P=(P_{ij})$ is \emph{bistochastic} if it satisfies (1)
$P_{ij}\geq0$, (2) $\sum_{j}P_{ij}=1$, and (3) $\sum_{i}P_{ij}=1$
for all $j$. Show that if $P$ is bistochastic, then $P^{2}$ is
also bistochastic. Use this to conclude, by induction, that $P^{n}$
is bistochastic.
\end{enumerate}
\item ~
\begin{enumerate}
\item Recall the matrix $P$ for the gambler's ruin with total wealth $N=4$
and probability of victory $p=1/2$:
\[
P=\begin{pmatrix}1 & 0\\
\nicefrac{1}{2} & 0 & \nicefrac{1}{2}\\
 & \nicefrac{1}{2} & 0 & \nicefrac{1}{2}\\
 &  & \nicefrac{1}{2} & 0 & \nicefrac{1}{2}\\
 &  &  & 0 & 1
\end{pmatrix}.
\]
Compute the eigenvalues of $P$. What is the multiplicity of the eigenvalue
$1$?
\item Now, consider a similar, but slightly different transition matrix:
\[
P^{\prime}=\begin{pmatrix}0 & 1\\
\nicefrac{1}{2} & 0 & \nicefrac{1}{2}\\
 & \nicefrac{1}{2} & 0 & \nicefrac{1}{2}\\
 &  & \nicefrac{1}{2} & 0 & \nicefrac{1}{2}\\
 &  &  & 0 & 1
\end{pmatrix}.
\]
We can interpret this as follows: when the gambler loses, the opponent
is kind and gives the gambler a dollar so that they can keep playing.
Compute the eigenvalues of $P^{\prime}$. What is the multiplicity
of the eigenvalue 1?
\item What do you think the multiplicity of the eigenvalue $1$ tells you
about the Markov chain?
\end{enumerate}
\end{enumerate}

\end{document}
