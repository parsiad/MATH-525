%% LyX 2.2.3 created this file.  For more info, see http://www.lyx.org/.
%% Do not edit unless you really know what you are doing.
\documentclass[english,12pt]{article}
\usepackage[T1]{fontenc}
\usepackage[latin9]{inputenc}
\usepackage{amssymb}

\makeatletter
%%%%%%%%%%%%%%%%%%%%%%%%%%%%%% User specified LaTeX commands.
\usepackage[margin=1in]{geometry}

\makeatother

\usepackage{babel}
\begin{document}

\title{Math 525: Assignment 6}

\date{\date{}}
\maketitle
\begin{enumerate}
\item Let $X$ be an absolutely continuous random variable. Show that if
its probability density is even (i.e., $f(x)=f(-x)$), then its characteristic
function must be real (i.e., $\phi(t)=\overline{\phi(t)}$).
\item Let $X_{n}\rightarrow X$ a.s. Show that $\phi_{n}\rightarrow\phi$
pointwise (here, $\phi_{n}$ and $\phi$ are the characteristic functions
of $X_{n}$ and $X$).
\item Let $X$ be a random variable and $\phi$ be its characteristic function.
Show that $|\phi|^{2}$ is also a characteristic function. You'll
need to use the following facts:
\begin{itemize}
\item For any complex number $z$, $|z|^{2}=z\overline{z}$.
\item You can construct a random variable $Y$ that has the same distribution
as $X$ but is independent of it\footnote{Rigorously, let $X$ be a random variable on the space $(\Omega,\mathcal{F},\mathbb{P})$.
Then, we can form the product space $(\Omega\times\Omega,\mathcal{F}\otimes\mathcal{F},\mathbb{P}\times\mathbb{P})$.
Denote a member of the new sample space $\Omega\times\Omega$ as $\omega\equiv(\omega_{1},\omega_{2})$.
Define $X_{1}(\omega_{1},\omega_{2})=X(\omega_{1})$ and $X_{2}(\omega_{1},\omega_{2})=X(\omega_{2})$
as two copies of the random variable $X$ that are independent.} so that $\mathbb{E}[f(X)]\mathbb{E}[g(Y)]=\mathbb{E}[f(X)g(Y)]$
for any Borel measurable functions $f$ and $g$.
\end{itemize}
\item Suppose $(X_{n})_{n}$ and $(Y_{n})_{n}$ converge in probability
to $X$ and $Y$, respectively. Show that $(X_{n}+Y_{n})_{n}$ converges
in distribution to $X+Y$ (hint: revisiting the proof for ``convergence
in probability implies convergence in distribution'' may help).
\end{enumerate}

\end{document}
